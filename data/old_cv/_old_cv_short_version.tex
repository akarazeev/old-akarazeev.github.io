\documentclass[margin, 9pt]{res} % Use the res.cls style, the font size can be changed to 11pt or 12pt here

\usepackage{xcolor}
\usepackage[colorlinks = true,
            linkcolor = blue,
            urlcolor  = blue,
            citecolor = blue,
            anchorcolor = blue]{hyperref}

\usepackage{helvet} % Default font is the helvetica postscript font
%\usepackage{newcent} % To change the default font to the new century schoolbook postscript font uncomment this line and comment the one above

\setlength{\textwidth}{5.1in} % Text width of the document

\newcommand\tab[1][1cm]{\hspace*{#1}}

\begin{document}

%----------------------------------------------------------------------------------------
%	NAME AND ADDRESS SECTION
%----------------------------------------------------------------------------------------

\moveleft.5\hoffset\centerline{\large\bf Anton Karazeev} % Your name at the top
\moveleft\hoffset\vbox{\hrule width\resumewidth height 1pt}\smallskip % Horizontal line after name; adjust line thickness by changing the '1pt'

\moveleft.5\hoffset\centerline{Moscow, Russia} % Your address
\moveleft.5\hoffset\centerline{+7-977-490-21-83 $\bullet$ anton.karazeev@phystech.edu}
\moveleft.5\hoffset\centerline{https://www.linkedin.com/in/akarazeev}

%$\bullet$ akarazeev.github.io}
%----------------------------------------------------------------------------------------
\begin{resume}

%----------------------------------------------------------------------------------------
%	EXPERIENCE SECTION
%----------------------------------------------------------------------------------------

\section{EXPERIENCE}

\textbf{\href{https://www.facebook.com/deepmipt/}{Laboratory of Neural Networks and Deep Learning}} \hfill October 2017 --- Present \\
{\sl Junior Researcher} \\
Currently responsible for preparing practical and theoretical assignments for the course of Reinforcement Learning and theoretical assignments for the course of Natural Language Processing with the number of 100+ enrolled students each.

\textbf{\href{http://www.generesearch.ru/Members.html}{Laboratory of Functional analysis of the Genome}} \hfill June 2016 --- Present \\
{\sl Research Assistant} \\
Research on protein function analysis.\\
Text mining, Natural language processing, Keyword extraction, Machine learning algorithms. As an intermediate result the new method of keywords extraction using Information Theory proposed (\href{https://www.researchgate.net/publication/319631846_Advanced_Parser_for_Biomedical_Texts_Poster_for_MCCMB\%2717}{ResearchGate}).

\textbf{Sberbank-Technology} \hfill August --- October 2017 \\
{\sl Data Scientist} \\
Responsible for Natural Language Processing projects. Participated in preparing the datasets and building baselines for competition \href{https://www.sdsj.ru/en/contest.html}{Sberbank Data Science Journey} which is based on \href{https://rajpurkar.github.io/SQuAD-explorer/}{SQuAD}. Developed an analogue of Amazon Mechanical Turk to improve experience of colleagues who evaluated the quality of collected datasets (Python, Flask).

\textbf{HiQE Group} \hfill March --- June 2017 \\
{\sl R\&D Data Scientist} \\
Negotiated with IBM engineers and applied some of the IBM Watson's services in tasks of signal processing. The system of baby cry recognition was built.

%----------------------------------------------------------------------------------------
%	EDUCATION SECTION
%----------------------------------------------------------------------------------------

\section{EDUCATION}
\textbf{Moscow Institute of Physics and Technology} \hfill 09.2014 --- 08.2018 (expected) \\
{\sl Department of Innovation and High Technologies, \\
Undergraduate student (B.Sc.)}
\begin{itemize}
\item Computer Science, Physics
\item Diploma - "Advanced toolkit for biomedical texts processing"
\end{itemize}

%----------------------------------------------------------------------------------------
%	PROJECTS SECTION
%----------------------------------------------------------------------------------------

\section{PROJECTS}

\textbf{Frontopolar}, Moscow  \hfill February - June, 2017 \\
Applied Reinforcement Learning for Stock Trading. State of the art results were reached.
\\Different approaches were tested including Q-learning and Recurrent Reinforcement Learning. References are listed \href{https://github.com/FRTP/Algorithms/wiki}{here}.

\textbf{Contributed to Open source:}
\begin{itemize}
	\item \href{https://github.com/RaRe-Technologies/gensim}{\textbf{Gensim}} - fixed issue \#671
	\item \href{https://github.com/yandexdataschool/Practical_RL}{\textbf{yandexdataschool/Practical\_RL}} - PR \#12
	\item \href{https://github.com/akarazeevprojects}{\textbf{My projects on GitHub}}
\end{itemize}

%----------------------------------------------------------------------------------------
%	SKILLS SECTION
%----------------------------------------------------------------------------------------

\section{SKILLS} 

\begin{itemize} 
\item \textbf{Russian}: native, \textbf{English}: fluent, \textbf{German}: basics (A2)
\item \textbf{Programming languages}: Python, C/C++, bash, R, experienced with SQL
\item \textbf{Python libraries}: numpy, sklearn, pandas; \textbf{for NLP}: NLTK, Gensim; \textbf{for Deep Learning}: TensorFlow, PyTorch
\item Experimented with RaspberryPi and Arduino. Projects are located on \href{https://github.com/akarazeevprojects}{GitHub}
\end{itemize}

%----------------------------------------------------------------------------------------
%	TEACHING SECTION
%----------------------------------------------------------------------------------------

\section{TEACHING}

\textbf{Deep Reinforcement Learning} \hfill October --- Present 2017\\
{\sl Course at MIPT, based on rll.berkeley.edu/deeprlcourse/}\\
\href{https://github.com/deepmipt/deep-rl-seminars}{Practical assignments}

\textbf{Deep Learning in Natural Language Processing} \hfill March --- Present 2017\\
{\sl Course at MIPT, based on cs224n.stanford.edu}\\
\href{https://github.com/deepmipt/deep-nlp-seminars}{Practical assignments}

%----------------------------------------------------------------------------------------
%	PUBLICATIONS SECTION
%----------------------------------------------------------------------------------------

\section{PUBLICATIONS}

\textbf{Medium Story} \hfill August, 2017 \\
"\href{https://blog.statsbot.co/generative-adversarial-networks-gans-engine-and-applications-f96291965b47}{Generative Adversarial Networks (GANs): Engine and Applications}"

\textbf{\href{http://mccmb.belozersky.msu.ru}{Moscow Conference on Computational Molecular Biology}} \hfill July 27 - 30, 2017 \\
Moscow, Russia
\\
"Advanced Parser for Biomedical Texts", \href{https://akarazeev.github.io/data/poster\_mccmb2017.pdf}{Poster}, \href{https://akarazeev.github.io/data/thesis\_mccmb2017.pdf}{Thesis}

%----------------------------------------------------------------------------------------
%	ADDITIONAL EDUCATION SECTION
%----------------------------------------------------------------------------------------

\section{ADDITIONAL EDUCATION}

\textbf{\href{http://deepbayes.ru}{Deep$\vert$Bayes Summer School}}, Moscow  \hfill August 26 - 30, 2017 \\
"Summer school on Bayesian Methods in Deep Learning"

\textbf{\href{http://bioinformaticsinstitute.ru/summer2017}{Bioinformatics Summer School}}, Moscow  \hfill July 31 - August 5, 2017 \\
"Big Data in Bioinformatics"

\textbf{Natural Language Processing} (based on cs224d.stanford.edu) \hfill 2016 \\
\tab {\sl by \href{http://info.deephack.me}{DeepHack Lab}} \\
\\
\textbf{Supercomputer technologies for atomistic modelling} \hfill 2015 \\
\tab {\sl by Igor Morozov (\href{http://www.ihed.ras.ru/norman/student/l-grid2.php}{IHED RAS})} \\
\href{https://github.com/akarazeev/MolecularDynamics-3sem-MIPT-2015}{Molecular Dynamics} - program written in C using OpenMP framework for parallel computing. Used \href{http://www.ks.uiuc.edu/Research/vmd/}{VMD} for visualization.

%----------------------------------------------------------------------------------------
%	HACKATHONS SECTION
%----------------------------------------------------------------------------------------

\section{HACKATHONS}

\textbf{\href{http://lauzhack.com}{LauzHack}}, EPFL, {\sl "NN:Nerds" team member} \hfill November 11 - 12, 2017 \\
1st place in challenge by \href{http://www.sgs.com}{SGS}, \href{https://akarazeev.github.io/data/Presentation_NNNerds.pdf}{Presentation}\\
Solution allows quick access to the main concepts found in documents. Responsible for development of telegram-bot (Python) and processing documents using IBM Watson service for Natural Language Understanding. \href{https://devpost.com/software/nn-nerds}{Devpost}.

\textbf{\href{https://mobility.abbyy.com/hack/}{mABBYYlity}}, Phystechpark  \hfill October 7 - 8, 2017 \\
4th place, "App in the Restaurant" iOS application, \href{https://youtu.be/YNgUMhhqIrs}{Demo}, \href{https://akarazeev.github.io/data/app_in_the_restaurant.pdf}{Presentation}\\
App allows to recognise entities from restaurant menus using smartphone's camera and translates them. ABBYY Real-Time Recognition SDK, ABBYY Lingvo API and Spoonacular API were used. Responsible for backend (Python).

\textbf{\href{http://neurocampus.webflow.io}{Neurocampus}}, Skolkovo Moscow School of Management  \hfill September 22 - 24, 2017 \\
\href{https://akarazeev.github.io/data/telegram_diploma_sos.PDF}{2nd place}, "S.o.S. - Sense of Speech" telegram-bot, \href{https://t.me/senseofspeech_bot}{@SenseOfSpeech\_bot}\\
Solution allows to extract emotions from user's recorded speech. Also it helps to train selected emotion with samples from TED talks.\\
Speech Emotion Recognition (SER) module was used as a core for telegram-bot based system to help users improve speech during performances. Responsible for development (Python).

\textbf{\href{http://bioinformaticsinstitute.ru/summer2017}{Bioinformatics Summer School}}, Moscow  \hfill August 3 - 4, 2017 \\
"Prediction of Experimental Metadata from Gene Expression" \\
Used Machine learning algorithms to predict phenotype by gene expression. \\
Distinguish with high accuracy samples of male and female tissues of \href{https://www.ncbi.nlm.nih.gov/Taxonomy/Browser/wwwtax.cgi?id=10090}{Mus musculus} organism. Datasets from Gene Expression Omnibus were used. \href{https://github.com/BioinfoGroup/MetaDataPredictionFromGE}{Project}.

\textbf{\href{http://biohack.ru}{BioHack 2017}}, Saint Petersburg \hfill March 3 - 5, 2017 \\
Text Mining, parsing the records from PubMed and UMLS.\\ Analysis of research trends of chemical compounds and diseases during period of 1990-2015 using parsed information from PubMed database. \href{https://github.com/akarazeev/BioHack2017}{Project}.


\end{resume}
\end{document}