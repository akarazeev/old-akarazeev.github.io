\documentclass[margin, 9pt]{res} % Use the res.cls style, the font size can be changed to 11pt or 12pt here

\usepackage{xcolor}
\usepackage[colorlinks = true,
            linkcolor = blue,
            urlcolor  = blue,
            citecolor = blue,
            anchorcolor = blue]{hyperref}

\usepackage{helvet} % Default font is the helvetica postscript font
%\usepackage{newcent} % To change the default font to the new century schoolbook postscript font uncomment this line and comment the one above

\setlength{\textwidth}{5.1in} % Text width of the document

\newcommand\tab[1][1cm]{\hspace*{#1}}

\begin{document}

%----------------------------------------------------------------------------------------
%	NAME AND ADDRESS SECTION
%----------------------------------------------------------------------------------------

\moveleft.5\hoffset\centerline{\large\bf Anton Karazeev} % Your name at the top
\moveleft\hoffset\vbox{\hrule width\resumewidth height 1pt}\smallskip % Horizontal line after name; adjust line thickness by changing the '1pt'

\moveleft.5\hoffset\centerline{Moscow, Russia} % Your address
\moveleft.5\hoffset\centerline{+7-977-490-21-83 $\bullet$ anton.karazeev@phystech.edu}
\moveleft.5\hoffset\centerline{https://www.linkedin.com/in/akarazeev}
\moveleft.5\hoffset\centerline{https://github.com/akarazeev/}

%$\bullet$ akarazeev.github.io}
%----------------------------------------------------------------------------------------
\begin{resume}

%----------------------------------------------------------------------------------------
%	POSITIONS SECTION
%----------------------------------------------------------------------------------------

\section{POSITIONS}

\textbf{Laboratory of Neural Networks and Deep Learning} \hfill October 2017 --- Present \\
{\sl Researcher}

\textbf{Laboratory of Functional analysis of the Genome} \hfill June 2016 --- Present \\
{\sl Research Assistant} \\
Research on protein function analysis.
\\
Text mining, Natural language processing, Keyword extraction, Machine learning \\algorithms (the \href{https://mipt.ru/science/labs/laboratoriya-funktsionalnogo-analiza-genoma/sotrudniki.php}{Laboratory} is located in \href{http://pharmcluster.ru/eng/}{BioPharmCluster} at MIPT, \\ \href{http://www.generesearch.ru/Members.html}{http://www.generesearch.ru/Members.html}).

\textbf{Sberbank-Technology} \hfill August --- October 2017 \\
{\sl Data Scientist} \\
Responsible for Natural Language Processing projects.

\textbf{Laboratory of Neural Networks and Deep Learning} \hfill March --- June 2017 \\
{\sl Teaching Assistant} \\
"Deep Learning in Natural Language Processing" course. Seminar quizzes preparing and home assignments checking (the \href{http://info.deephack.me}{Laboratory} is located in \href{http://pharmcluster.ru/eng/}{BioPharmCluster} at MIPT).

\textbf{HiQE Group} \hfill March --- June 2017 \\
{\sl R\&D Data Scientist} \\
Responsible for: negotiations with IBM engineers to determine which IBM tools including Watson services are useful to HiQE Group's projects, audio signal processing using machine learning methods.


%----------------------------------------------------------------------------------------
%	TEACHING SECTION
%----------------------------------------------------------------------------------------

\section{TEACHING}

\textbf{Deep Reinforcement Learning} \hfill October --- Present 2017\\
{\sl Course at MIPT, based on rll.berkeley.edu/deeprlcourse/}\\
Practical assignments: \href{https://github.com/deepmipt/deep-rl-seminars}{https://github.com/deepm ipt/deep-rl-seminars}

\textbf{Deep Learning in Natural Language Processing} \hfill March --- Present 2017\\
{\sl Course at MIPT, based on cs224n.stanford.edu}\\
Practical assignments: \href{https://github.com/deepmipt/deep-nlp-seminars}{https://github.com/deepmipt/deep-nlp-seminars}

%----------------------------------------------------------------------------------------
%	EDUCATION SECTION
%----------------------------------------------------------------------------------------

\section{EDUCATION}
\textbf{Moscow Institute of Physics and Technology} \hfill 09.2014 --- 08.2018 (expected) \\
{\sl Department of Innovation and High Technologies, \\
Undergraduate student (B.Sc.)}
\begin{itemize}
\item Discrete mathematics: mathematical logic, discrete analysis, probability theory, mathematical statistics (with practical assignments in Jupyter notebooks), random processes
\item Mathematics: mathematical analysis, linear algebra, differential equations, computational mathematics
\item Computer science: Python, SQL, C/C++ with OpenMP and MPI frameworks, algorithms and data structures, OOP and design patterns, multithreading and concurrency, distributed computing, MapReduce, Hadoop and Hive
\item General physics (lectures, seminars, lab work), theoretical physics
\item Other: business communications, chemistry
\end{itemize}

%----------------------------------------------------------------------------------------
%	ADDITIONAL EDUCATION SECTION
%----------------------------------------------------------------------------------------

\section{ADDITIONAL EDUCATION}

\textbf{Deep$\vert$Bayes Summer School}, Moscow  \hfill August 26 - 30, 2017 \\
"Summer school on Bayesian Methods in Deep Learning" \\
\href{http://deepbayes.ru}{http://deepbayes.ru}

\textbf{Bioinformatics Summer School}, Moscow  \hfill July 31 - August 5, 2017 \\
"Big Data in Bioinformatics" \\
\href{http://bioinformaticsinstitute.ru/summer2017}{http://bioinformaticsinstitute.ru/summer2017}

\textbf{Natural Language Processing} (based on cs224d.stanford.edu) \hfill 2016 \\
\tab {\sl by DeepHack Lab} \\
\\
\textbf{Supercomputer technologies for atomistic modelling} \hfill 2015 \\
\tab {\sl by Igor Morozov (\href{http://www.ihed.ras.ru/norman/student/l-grid2.php}{IHED RAS})} \\
Molecular Dynamics - program written in C using OpenMP framework \\ for parallel computing. Used VMD for visualization. Code on GitHub - \\
\href{https://github.com/akarazeev/MolecularDynamics-3sem-MIPT-2015}{https://github.com/akarazeev/MolecularDynamics-3sem-MIPT-2015}.

%----------------------------------------------------------------------------------------
%	CONFERENCES SECTION
%----------------------------------------------------------------------------------------

\section{CONFERENCES}

\textbf{Moscow Conference on Computational Molecular Biology} \hfill July 27 - 30, 2017 \\
Moscow, Russia
\\
"Advanced Parser for Biomedical Texts" \\
Poster: \href{https://akarazeev.github.io/data/poster\_mccmb2017.pdf}{https://akarazeev.github.io/data/poster\_mccmb2017.pdf} \\
Thesis: \href{https://akarazeev.github.io/data/thesis\_mccmb2017.pdf}{https://akarazeev.github.io/data/thesis\_mccmb2017.pdf}

%----------------------------------------------------------------------------------------
%	PUBLICATIONS SECTION
%----------------------------------------------------------------------------------------

\section{PUBLICATIONS}

\textbf{Medium Story} \hfill August, 2017 \\
\href{https://blog.statsbot.co/generative-adversarial-networks-gans-engine-and-applications-f96291965b47}{"Generative Adversarial Networks (GANs): Engine and Applications"}

%----------------------------------------------------------------------------------------
%	HACKATHONS SECTION
%----------------------------------------------------------------------------------------

\section{HACKATHONS}

\textbf{LauzHack}, EPFL, {\sl "NN:Nerds" team member} \hfill November 11 - 12, 2017 \\
1st place in challenge by \href{http://www.sgs.com}{SGS}, \href{https://akarazeev.github.io/data/Presentation_NNNerds.pdf}{presentation}\\
Responsible for development of telegram-bot (Python) and processing documents using IBM Watson service for Natural Language Understanding.

\textbf{mABBYYlity}, Phystechpark  \hfill October 7 - 8, 2017 \\
"App in the Restaurant" iOS application, \href{https://youtu.be/YNgUMhhqIrs}{demo}, \href{https://akarazeev.github.io/data/app_in_the_restaurant.pdf}{presentation}\\
App allows to recognise entities from restaurant menus using smartphone's camera and translates them. ABBYY Real-Time Recognition SDK, ABBYY Lingvo API and Spoonacular API were used. Responsible for backend (Python).

\textbf{Neurocampus}, Skolkovo Moscow School of Management  \hfill September 22 - 24, 2017 \\
"S.o.S. - Sense of Speech" telegram-bot, \href{https://t.me/senseofspeech_bot}{@SenseOfSpeech\_bot}, \href{https://akarazeev.github.io/data/telegram_diploma_sos.PDF}{2nd place}\\
Speech Emotion Recognition (SER) module was used as a core for telegram-bot based system to help users improve speech during performances. Responsible for development (Python).

\textbf{Bioinformatics Summer School}, Moscow  \hfill August 3 - 4, 2017 \\
"Prediction of Experimental Metadata from Gene Expression" \\
Used Machine learning algorithms to predict phenotype by gene expression. Datasets from Gene Expression Omnibus were used. Link to project on GitHub - \\ \href{https://github.com/BioinfoGroup/MetaDataPredictionFromGE}{https://github.com/BioinfoGroup/MetaDataPredictionFromGE}

\textbf{BioHack 2017}, Saint Petersburg \hfill March 3 - 5, 2017 \\
Text Mining, parsing the records from PubMed and UMLS. \\
Link to project on GitHub - \href{https://github.com/akarazeev/BioHack2017}{https://github.com/akarazeev/BioHack2017}

\textbf{Junction 2016}, Helsinki, {\sl "Dreamteam" team member} \hfill November 25 - 27, 2016 \\
Used a python wrapper around the Twitter API and Topic Modeling of tweets (gensim).

%----------------------------------------------------------------------------------------
%	PROJECTS SECTION
%----------------------------------------------------------------------------------------

\section{PROJECTS}

\textbf{Frontopolar}, Moscow  \hfill February - June, 2017 \\
Applied Reinforcement Learning for Stock Trading. \\Different approaches were tested including Q-learning and Recurrent Reinforcement Learning. Link to project's wiki on GitHub - \href{https://github.com/FRTP/Algorithms/wiki}{https://github.com/FRTP/Algorithms/wiki}

\textbf{Contributed to Open source:}
\begin{itemize}
	\item \href{https://github.com/akarazeevprojects}{\textbf{https://github.com/akarazeevprojects}}
	\item \href{https://github.com/RaRe-Technologies/gensim}{\textbf{Gensim}} - fixed issue \#671
	\item \href{https://github.com/yandexdataschool/Practical_RL}{\textbf{yandexdataschool/Practical\_RL}} - PR \#12
\end{itemize}

%----------------------------------------------------------------------------------------
%	MOOC SECTION
%----------------------------------------------------------------------------------------

\section{MOOC}

\textbf{Machine Learning} \hfill 2016 \\
\tab {\sl by Yandex \& MIPT on coursera.org} \\
\textbf{Neural Networks} \\
\tab {\sl by Bioinformatics Institute on stepik.org} \\
\textbf{Molecular Biology and Genetics} \\
\tab {\sl by Bioinformatics Institute on stepik.org} \\
\\
\textbf{Discrete Structures} \hfill 2015 \\
\tab {\sl by Alex Dainiak (MIPT) on stepik.org}

%----------------------------------------------------------------------------------------


%----------------------------------------------------------------------------------------
%	Technology SKILLS SECTION
%----------------------------------------------------------------------------------------

\section{SKILLS} 

\begin{itemize} 
\item \textbf{Russian}: native, \textbf{English}: fluent, \textbf{German}: basics (A2)
\item Started "MIPT Deep Learning Club" (\href{https://www.facebook.com/dlmipt}{https://www.facebook.com/dlmipt}, \\ \href{https://vk.com/dlmipt}{https://vk.com/dlmipt}) to discuss and share ideas on deep learning topics. Led a few seminars on topics such as "Introduction to bayesian methods". Recordings can be found here: \href{https://www.youtube.com/playlist?list=PLsGmQ_AGA4yxVRTFrpnpsXrYSLADxGmTO}{youtube.com} and \href{https://www.pscp.tv/akarazeev}{pscp.tv}
\item Experimented with RaspberryPi and Arduino. Projects are located on GitHub (\href{https://github.com/akarazeev}{https://github.com/akarazeev} and \href{https://github.com/akarazeevprojects}{https://github.com/akarazeevprojects})
\end{itemize}

\end{resume}
\end{document}